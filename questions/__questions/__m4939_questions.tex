% Options for packages loaded elsewhere
\PassOptionsToPackage{unicode}{hyperref}
\PassOptionsToPackage{hyphens}{url}
%
\documentclass[
]{article}
\usepackage{lmodern}
\usepackage{amssymb,amsmath}
\usepackage{ifxetex,ifluatex}
\ifnum 0\ifxetex 1\fi\ifluatex 1\fi=0 % if pdftex
  \usepackage[T1]{fontenc}
  \usepackage[utf8]{inputenc}
  \usepackage{textcomp} % provide euro and other symbols
\else % if luatex or xetex
  \usepackage{unicode-math}
  \defaultfontfeatures{Scale=MatchLowercase}
  \defaultfontfeatures[\rmfamily]{Ligatures=TeX,Scale=1}
\fi
% Use upquote if available, for straight quotes in verbatim environments
\IfFileExists{upquote.sty}{\usepackage{upquote}}{}
\IfFileExists{microtype.sty}{% use microtype if available
  \usepackage[]{microtype}
  \UseMicrotypeSet[protrusion]{basicmath} % disable protrusion for tt fonts
}{}
\makeatletter
\@ifundefined{KOMAClassName}{% if non-KOMA class
  \IfFileExists{parskip.sty}{%
    \usepackage{parskip}
  }{% else
    \setlength{\parindent}{0pt}
    \setlength{\parskip}{6pt plus 2pt minus 1pt}}
}{% if KOMA class
  \KOMAoptions{parskip=half}}
\makeatother
\usepackage{xcolor}
\IfFileExists{xurl.sty}{\usepackage{xurl}}{} % add URL line breaks if available
\IfFileExists{bookmark.sty}{\usepackage{bookmark}}{\usepackage{hyperref}}
\hypersetup{
  pdftitle={MATH 4939 Questions},
  hidelinks,
  pdfcreator={LaTeX via pandoc}}
\urlstyle{same} % disable monospaced font for URLs
\usepackage[margin=1in]{geometry}
\usepackage{graphicx,grffile}
\makeatletter
\def\maxwidth{\ifdim\Gin@nat@width>\linewidth\linewidth\else\Gin@nat@width\fi}
\def\maxheight{\ifdim\Gin@nat@height>\textheight\textheight\else\Gin@nat@height\fi}
\makeatother
% Scale images if necessary, so that they will not overflow the page
% margins by default, and it is still possible to overwrite the defaults
% using explicit options in \includegraphics[width, height, ...]{}
\setkeys{Gin}{width=\maxwidth,height=\maxheight,keepaspectratio}
% Set default figure placement to htbp
\makeatletter
\def\fps@figure{htbp}
\makeatother
\setlength{\emergencystretch}{3em} % prevent overfull lines
\providecommand{\tightlist}{%
  \setlength{\itemsep}{0pt}\setlength{\parskip}{0pt}}
\setcounter{secnumdepth}{-\maxdimen} % remove section numbering

\title{MATH 4939 Questions}
\author{}
\date{\vspace{-2.5em}}

\begin{document}
\maketitle

\begin{enumerate}
\def\labelenumi{\arabic{enumi}.}
\item
  In order to assess factors related to reckless driving behaviors,
  investigators ran a study in which observers at different
  intersections with a stop sign recorded the number of cars that did
  not stop properly along with various information on each violator,
  including gender of the driver, type of car (sedan, sports utility,
  mini-van, wagon, truck, other), and approximate age of the driver
  (under 30, 30-40, 40-50, 50+). Pooling information from several
  intersections and for different observers, investigators recorded the
  number of violators in each category. \newline  Describe a generalized
  linear model for analyzing these data (specify a reasonable, possibly
  non-linear, model and its transformation to linear form as well as a
  conditional distribution and link function) and outline a specific
  approach for assessing the following questions of interest using
  frequentist methods: (1) is gender an important predictor? (2) is type
  of car an important predictor, and if so which types are predictive of
  a greater frequency of violations? (3) is there a trend with age in
  the frequency of violations? \vspace{0px}
\item
  In marketing research, it is widely believed that subliminal messages
  in advertisements can be effective in improving a consumer's
  impression of a product. To test whether the type and frequency of the
  subliminal message has an impact, investigators ran a study in which
  they enrolled male and female graduate students. Study subjects were
  all shown an outwardly-identical taped advertisement for a soft drink
  and then asked to rank their impression of the soft drink after
  watching the tape on a scale from 1-5 (1=strongly negative, 2=mildly
  negative, 3=no opinion, 4=mildly positive, 5=strongly positive). Tapes
  varied in the type of subliminal message (1=none, 2=attractive female
  face, 3=attractive male face) and (for tapes with a subliminal
  message) the frequency (1=low, 2=medium, 3=high). \newline Describe a
  regression model for relating gender, type of subliminal message, and
  frequency to an individual's impression. Describe the specifics of a
  Bayesian approach for addressing the following questions of interest
  to the investigator: (1) do subliminal messages have an effect
  overall? (2) does this effect vary depending on the gender of the
  observer? (3) does the effect depend on the frequency? and (4) do
  males and females respond differently depending on the type of
  subliminal message? Detail the prior used, the form of the regression
  model, and the likelihood. Provide an outline of the method used for
  posterior computation and inferences on the above questions of
  interest. \vspace{0px}
\item
  Suppose that a random variable \(Y\) has a Poisson distribution with
  mean \(\lambda\) were
  \(ln(\lambda) = \beta_0 + \beta_1 X_1 + \beta_2 X_2\). Assuming that
  \(X_2\) is not constant and that \(\beta_2 \ne 0\), show that \(Y\)
  does not have a Poisson distribution if one fits a model using only
  \(X_1\). \vspace{0px}
\item
  Uterine fibroids are a common reproductive tract tumor. To study
  factors related to fibroid incidence (that is, the rate of onset for
  women who do not have fibroids), women aged 20-30 who did not have
  fibroids at a baseline examination were enrolled in a prospective
  study. These women were then given a screening examination
  approximately every 5 years (though the specific ages at examination
  varied) to assess whether fibroids had yet developed. Each woman was
  followed for 15 years (3 examinations) or until she either developed
  fibroids or dropped out of the study. Various information was
  collected for each women, including race (white, black, other), age at
  menarche, age at entry into the study, age at each examination, and
  whether the woman was a smoker. \newline Assuming that fibroids do not
  go away once they have developed, describe a regression model for
  these data that allows fibroid incidence to vary with age and other
  factors. Show the observed data likelihood under this regression
  model, and develop a Bayesian approach for estimating age-specific
  fibroid incidence for women in different groups. Detail the prior used
  and the form of the posterior. Outline an approach for posterior
  computation and describe specifically how you obtain point and
  interval estimates for the probability of developing fibroids by a
  given age for women in different groups. \vspace{0px}
\item
  For adults who sailed on the Titanic on its fateful voyage, the odds
  ratio between gender (female, male) and survival (yes, no) was 11.4.
  What is wrong with the interpretation, ``The probability of survival
  for females was 11.4 times that for males.''
\end{enumerate}

\vspace{0px}

\begin{enumerate}
\def\labelenumi{\arabic{enumi}.}
\setcounter{enumi}{5}
\tightlist
\item
  \emph{(continued from previous question)} When would the quoted
  interpretation be approximately correct? Why?
\end{enumerate}

\vspace{0px}

\begin{enumerate}
\def\labelenumi{\arabic{enumi}.}
\setcounter{enumi}{6}
\item
  \emph{(continued from previous question)} The odds of survival for
  females equaled 2.9. For each gender, find the proportion who
  survived. \vspace{0px}
\item
  Explain what is meant by overdispersion, and explain how it can occur
  for Poisson generalized linear models for count data. \vspace{0px}
\item
  Explain two ways in which the generalized linear model extends the
  ordinary regression model that is commonly used for quantitative
  response variables. \vspace{0px}
\item
  Each of 100 multiple-choice questions on an exam has five possible
  answers but one correct response. For each question, a student
  randomly selects one response as the answer. Specify the probability
  distribution of the student's number of correct answers on the exam,
  identifying the parameter(s) for that distribution. Would it be
  surprising if the student made at least 50 correct responses? Explain
  your reasoning. \vspace{0px}
\item
  Suppose \texttt{y}, \texttt{x}, and \texttt{z} are numerical variables
  in the R data frame \texttt{dd}. \newline Explain the difference
  between a linear model fitted with the formula
  \texttt{y\ \textasciitilde{}\ x*z} in comparison with the formula
  \texttt{y\ \textasciitilde{}\ I(x*z)}. \vspace{0px}
\item
  There is a lab test for a rare disease D that has a specificity of .98
  and a sensitivity of .95. Suppose the prevalence of the disease is
  .01\%. Explain how, if one thinks of the lab test as a hypothesis test
  for the null hypothesis of no disease, a positive result produces a
  p-value of 0.01. \vspace{0px}
\item
  \emph{(continued from previous question)} If someone selected at
  random (for example if the test is used to screen for the disease) is
  given the test and gets a positive result, what is the probability
  that they have the disease? \vspace{0px}
\item
  An article states that the PSA blood test for detecting prostate
  cancer stated that, of men who had this disease, the test fails to
  detect prostate cancer in 1 in 4, and, of men who do not have it,
  approximately 2/3 received false positive results. Let \(D\) (\(D^c\))
  denote the event of having (not having) prostate cancer and let
  \(Pos\) (\(Neg\)) denote a positive (negative) test result. What is
  the sensitivity and specificity of this test? (from Agresti, 2007)
  \vspace{0px}
\item
  \emph{(continued from previous question)} Of men who take the PSA
  test, 1\% have the disease. Find the cell probabilities in the
  \(2 \times 2\) table for the joint distribution for having (not
  having) the disease versus a positive or negative test result.
  \vspace{0px}
\item
  \emph{(continued from previous question)} Find the probability of
  having prostate cancer given a positive test result and find the
  probability of not having the disease given a negative test result.
  \vspace{0px}
\item
  A lab test for a rare disease D has a specificity of .95 and a
  sensitivity of .95. Suppose the prevalence of the disease is .01\%.
  What proportion of the time is the test in error? \vspace{0px}
\item
  \emph{(continued from previous question)} Given a positive test
  result, what is probability of error? \vspace{0px}
\item
  \emph{(continued from previous question)} Given a negative test
  result, what is probability of error? \vspace{0px}
\item
  \emph{(continued from previous question)} How do you reconcile the 3
  preceding results? \vspace{0px}
\item
  A British study in 1998 report that, of smokers who get lung cancer,
  ``women were 1.7 times more vulnerable than men to get small-cell lung
  cancer''. What kind of statistic is the figure `1.7' reported here?
  (Agresti, 2007) \vspace{0px}
\item
  A National Cancer Institute study about tamoxifen and breast cancer
  reported in 1998 that the women taking the drug were 45\% less likely
  to experience invasive breast cancer compared with the women taking
  placebo. Find the relative risk for (i) those taking the drug compared
  to those taking placebo, (ii) those taking placebo compared to those
  taking the drug. \vspace{0px}
\item
  In the United States, data reported in 1993 indicates that the annual
  probability that a woman over the age of 35 dies of lung cancer equals
  0.001304 for current smokers and 0.000121 for nonsmokers. Calculate
  and interpret the difference of proportions, the relative risk and the
  odds ratio. Which is more informative? Why? (Agresti, 2007)
  \vspace{0px}
\item
  \emph{(continued from previous question)} Are the odds ratio and
  relative risk similar or dissimilar? Why? \vspace{0px}
\item
  With a \(2 \times 2\) table what is the effect of interchanging two
  rows on the odds ratio? On the log-odds ratio? \vspace{0px}
\item
  AVP: Find an approximate 95\% confidence intererval for
  \(\beta_{XXXX}\). Show how you obtained it. You don't need to describe
  the process in detail but show evidence of the process you used.
  \vspace{0px}
\item
  \emph{(continued from previous question)} Anova anova show the null
  model and the alternative model for each line. You may use R's linear
  model formula notation to define the models. \vspace{0px}
\item
  \emph{(continued from previous question)} Linear combination question.
  \vspace{0px}
\item
  The eagle question with the gamma. Include information on the gamma
  \newpage \vspace{0px}
\item
  23Q \vspace{0px}
\item
  23Q \vspace{0px}
\item
  Beta space question \vspace{0px}
\item
  The following is output from an script in R using the Prestige data in
  the `car' package: \newline \includegraphics{m4330/q22a.PNG}
  \newline and this is the plot generated by the script:
  \newline \includegraphics{m4330/q22b.PNG} \newline If you have the
  information to construct an approximate 95\% confidence interval for
  the coefficient of `education' controlling for `type', do so. Show how
  you constructed the interval. You do not need to explain how you
  constructed it in detail as long as it is obvious from your work on
  the exam. If you do not have the information, explain why not.
  \vspace{0px}
\item
  \emph{(continued from previous question)} What is the difference in
  predicted income from this model for someone with 12 years of
  education in a professional occupation versus a white collar
  occupation? \vspace{0px}
\item
  Consider the problem of imputing a grade for a missed mid-term mark
  from a student's performance on a final exam. Suppose a professor is
  considering three possibilities: 1) Use the student's z-score on the
  final exam to impute the z-score on the mid-term, 2) perform a linear
  regression of mid-term grades on the final exam grade and impute the
  mid-term grade by using the predicted mid-term grade, and 3) use a
  linear regression of the final exam grade on the mid-term grade and
  impute the mid-term grade that would have predicted the student's
  final exam grade. \newline Which of these three methods would be
  advantageous for a student who has a very high grade on the final
  exam? Which would be advantageous for a student who has a very low
  grade on the final exam? Clearly show the reasoning behind your
  answer. \vspace{0px}
\item
  Explain what is meant by the Hauck-Donner phenomenon. How does it
  affect the practice of logistic regression? \vspace{0px}
\item
  In the early stages of an epidemic, the number of new cases grows
  exponentially. Suppose that the expected number of cases on day
  \(t_i\) is modelled as \(\gamma e^{\delta t_i}\). Specify a GLM that
  you could use to analyze data in which the response is the number of
  new cases on a number of specific days, for example: days 5, 6, 10,
  12, 20 and 23. Describe in detail the transformation of the model to
  linear form and the interpretation of the original parameters above
  and of the linear parameters of the model. \vspace{0px}
\item
  Let \(Y\) be number of fatal accidents on a day in Toronto. Suppose
  the expected number of fatal car accidents depends on a number of
  variables and that, if all of these variables were taken into account,
  the conditional distribution of \(Y\) would be Poisson. Show that,
  assuming that the variables are not all constant, the unconditional
  distribution of \(Y\) itself is overdispersed relative to that of a
  Poisson distribution. \vspace{0px}
\item
  Give a real world example of three variables, \(X, Y,\) and \(Z\), for
  which we would expect \(X\) and \(Y\) to be marginally associated but
  conditionally independent controlling for \(Z\). \vspace{0px}
\item
  The~Cowles~data frame has 1421 rows and 4 columns. These data come
  from a study of the personality determinants of volunteering for
  psychological research. Neuroticism (neuro) is classified in three
  levels: low, medium and high. Extraversion (extra) is measured on a
  scale that ranges from 1 to 25. The purpose of the study is to explore
  some personality predictors of the willingness to volunteer and how
  the prediction differs between men and women. This is some output from
  an R script: \newline \includegraphics{m4330/q25a.PNG} \newline What
  is the predicted probability of volunteering for a male with high
  `neuro' and extraversion equal to 20? \vspace{0px}
\item
  \emph{(continued from previous question)} For each row of the
  following Anova table specify both the null hypothesis and the
  alternative hypothesis being tested. You may use R's linear model
  formulas to express the hypotheses or you may use the parameters of
  the model provided they are named in a way that is
  clear.\newline \includegraphics{m4330/q25b.PNG} \newline \vspace{0px}
\item
  \emph{(continued from previous question)} The following is an effect
  plot produced by the `effects' package.
  \newline \includegraphics{m4330/q25c.PNG} \newline  What are the main
  reasons that the confidence bands are wider for some combinations of
  neuro and sex than for others? \vspace{0px}
\item
  Sally Clark was convicted of murder of her two children who died with
  no signs of illness or trauma on the strength of statistical evidence
  that claimed that assuming the `null hypothesis' that she is innocent,
  the probability of 2 deaths occuring due to the only known
  explanation, sudden infant death syndrome, was extremely small. In
  testimony, Sir Roy Meadow claimed the probability to be 1 in 73
  million but better estimates would put it at a maximum of 1 in
  100,000.

  \par

  Explain whether 0.00001 can be considered a p-value for assessing the
  hypothesis that Sally Clark is innocent. Discuss whether there are
  other ways of assessing the probability of her guilt? Work out a rough
  calculation based on reasonable guesses for the result of such a
  calculation. \vspace{0px}
\item
  Discuss how it could be possible for a regression with two linear
  predictors to produce two different final models when using forward
  stepwise versus backward stepwise variable selection algorithms.
  Explain how a confidence region for the coefficients of the two linear
  predictors is related to forward and backward stepwise selection
  \vspace{0px}
\item
  A survey of students taking a large course showed (this is true) that
  students who viewed the recorded videos of the lectures many times
  performed less well on the final exam than students who viewed the
  videos fewer times. Upon discovering this study, your professor
  announces that they will discontinue recording the lectures because,
  the professor says, there is evidence that the videos cause students
  to do perform more poorly on courses. Comment on the professor's
  reasoning using concepts we have studied in this course. \vspace{0px}
\item
  Choose a possible confounding factor and use a Paik-Agresti diagram to
  show how controlling for this confounding factor could reverse the
  direction of association between the frequency of video viewing and
  performance on the course. \vspace{0px}
\item
  What output will the following R script produce? Explain briefly why.

  \vspace{0px}
\item
  What output will the following R script produce? Explain briefly why.

  \vspace{0px}
\item
  Let x be defined as:

  Write an R function that would turn \texttt{x} into a factor whose
  ordering corresponds to the numerical ordering of \texttt{x}.
  \vspace{0px}
\item
  In R, let \texttt{x\ \textless{}-\ 1:5}. What output would
  \texttt{x{[}NA{]}} produce? What output would
  \texttt{x{[}NA\_real\_{]}} produce? Describe the reason for the
  difference, if any. \vspace{0px}
\item
  In R, describe the result of subsetting a vector with positive
  integers, with negative integers, with a logical vector, or with a
  character vector? \vspace{0px}
\item
  In R, what's the difference between~{[},~{[}{[}, and~\$~when applied
  to a list? \vspace{0px}
\item
  In R, when subsetting with {[}, when should you use~drop = FALSE?
  Include arrays and factors in your discussion. \vspace{0px}
\item
  In R, If~\texttt{x}~is a matrix, what
  does~\texttt{x{[}{]}\ \textless{}-\ 0}~do? How is it different
  from~\texttt{x\ \textless{}-\ 0}? \vspace{0px}
\item
  In R, how can you use a named vector to relabel a categorical
  variable? \vspace{0px}
\item
  In R, the data frame \texttt{mtcars} has 32 rows and 11 variables of
  which \texttt{cyl} is a variable recording the number of cylinders in
  each type of car. Fix each of the following common data frame
  subsetting errors in R: \newline\newline
\end{enumerate}

\texttt{mtcars{[}mtcars\$cyl\ =\ 4,\ {]}}\newline\newline\newline\newline
\texttt{mtcars{[}-1:4,\ {]}}\newline\newline\newline\newline
\texttt{mtcars{[}mtcars\$cyl\ \textless{}=\ 5{]}}\newline\newline\newline\newline
\texttt{mtcars{[}mtcars\$cyl\ ==\ 4\ \textbar{}\ 6,\ {]}}\newline\newline\newline\newline

\vspace{0px}

\begin{enumerate}
\def\labelenumi{\arabic{enumi}.}
\setcounter{enumi}{56}
\item
  In R, if mtcars is a data frame, why does mtcars{[}1:20{]} return an
  error? How does it differ from the similar mtcars{[}1:20, {]}?
  \vspace{0px}
\item
  In R, if df is a data frame, what does df{[}is.na(df){]} \textless- 0
  do? How does it work? \vspace{0px}
\item
  Create the vector (20,19, \ldots{} ,2,1) in R. \vspace{0px}
\item
  Create the vector (1,2,3, \ldots{} ,19,20,19,18, \ldots{} ,2,1) in R.
  \vspace{0px}
\item
  Create the vector (4,4, \ldots{} ,4,6,6, \ldots{} ,6,3,3, \ldots{} ,3)
  in R, where there are 10 occurrences of 4, 20 of 6 and 30 of 3.
  \vspace{0px}
\item
  Write an expression in R to calculate the following
  \(\sum_{i=10}^{100} (i^3 + 4i^2)\) \vspace{0px}
\item
  Generate in R a vector of 30 labels: `label 1', `label 2', \ldots{}
  `label 30' \vspace{0px}
\item
  Let \texttt{y\ \textless{}-\ sample(1000,\ 30,\ replace\ =\ TRUE)}.
  Write an expression in R to determine how many elements of \texttt{y}
  are divisible by 2. \vspace{0px}
\item
  Let \texttt{y\ \textless{}-\ sample(1000,\ 30,\ replace\ =\ TRUE)}.
  Write an expression in R to determine how many elements of \texttt{y}
  are divisible by 2. \vspace{0px}
\item
  Let \texttt{y\ \textless{}-\ sample(1000,\ 30,\ replace\ =\ TRUE)}.
  Write an expression in R to determine how many elements of \texttt{y}
  are within 200 of the maximum value. \vspace{0px}
\item
  Let \texttt{y\ \textless{}-\ sample(1000,\ 30,\ replace\ =\ TRUE)}.
  Write an expression in R to determine how many elements of \texttt{y}
  are less than the previous element. \vspace{0px}
\item
  Let \texttt{y\ \textless{}-\ sample(1000,\ 30,\ replace\ =\ TRUE)}.
  Write an expression in R to determine how many elements of \texttt{y}
  are an exact square. \vspace{0px}
\item
  Suppose data for a variable in R representing dollars has been entered
  in a variety of formats: `\$1,000.00',`1000.00',`\$1'. Write a
  function in R that transforms the variable to a numeric variable in
  dollars to the nearest cent. \vspace{0px}
\item
  Write a function in R that takes a character string and collapses
  multiple adjoining blanks to a single blank. \vspace{0px}
\item
  Use the site Gapminder.org to download at least three longitudinal
  variables into separate data sets. Merge the data sets into one for
  which each row represents one country and year and contains the values
  of each of the three variables you downloaded. \vspace{0px}
\item
  Write a function in R that removes from a data frame every variable
  whose name starts with the letter `X' and ends in a number.
  \vspace{0px}
\item
  Write R code to create a 6 by 10 matrix of random integers in R as
  follows:

  Write a function to find the number of entries in each row that are
  greater than 4. \vspace{0px}
\item
  Let \texttt{mat} be a matrix of integers in R. Write a function to
  find how many rows have exactly two instances of the number 7.
  \vspace{0px}
\item
  Describe the difference in R between
  \texttt{paste(x,\ y,\ sep\ =\ \textquotesingle{}:\textquotesingle{})}
  and
  \texttt{paste(x,\ y,\ collapse\ =\ \textquotesingle{}:\textquotesingle{})}
  \vspace{0px}
\item
  Using the \texttt{hs} data set in the spida2 package, create a plot
  with two panels showing histograms displaying the distribution of
  school sizes in the Public and in the Catholic sectors. Use the
  functions \texttt{capply} and \texttt{up} in the spida2 package. You
  may also use any other approach to compare with the use of
  \texttt{capply} and \texttt{up}. \vspace{0px}
\item
  Using the \texttt{hs} data set in the spida2 package, create a plot
  with two panels showing histograms displaying the distribution of
  sample sizes in each school in the Public and in the Catholic sectors.
  Use the functions \texttt{capply} and \texttt{up} in the spida2
  package. You may also use any other approach to compare with the use
  of \texttt{capply} and \texttt{up}. \vspace{0px}
\item
  Using the \texttt{hs} data set in the spida2 package, create a plot
  with two panels showing scatterplots displaying the relationship
  between mean mathach and mean ses in each school in the Public and in
  the Catholic sectors. Explore reasonable transformations and
  regression lines: linear and non-parametric in the plots. Use the
  functions \texttt{capply} and \texttt{up} in the spida2 package. You
  may also use any other approach to compare with the use of
  \texttt{capply} and \texttt{up}. \vspace{0px}
\item
  Describe the difference in R between a `generic function' and a
  method. \vspace{0px}
\item
  What is wrong with the following claim: ``Data show that income and
  marriage have a high positive correlation. Therefore, your earnings
  will increase if you get married.'' \vspace{0px}
\item
  What is wrong with the following claim: ``Data show that as the number
  of fires increase, so does the number of fire fighters. Therefore, to
  cut down on fires, you should reduce the number of fire fighters.''
  \vspace{0px}
\item
  What is wrong with the following claim: ``Data show that people who
  hurry tend to be late to their meetings. Don't hurry, or you'll be
  late.'' \vspace{0px}
\item
  A baseball batter Tim has a better batting average than his teammate
  Frank. However, someone notices that Frank has a better batting
  average than Tim against both right-handed and left-handed pitchers.
  How can this happen? Present your answer as a hypothetical table and
  in a Paik diagram. \vspace{0px}
\item
  Discuss whether you should use the aggregate (marginal) or the
  segregated (conditional) data to attempt to determine the true effect
  in the following situation: There are two treatments used on kidney
  stones: Treatment A and Treatment B. Doctors are more likely to use
  Treatment A on large (and therefore, more severe) stones and more
  likely to use Treatment B on small stones. Should a patient who
  doesn't know the size of his or her stone examine the general
  population data, or the stone size-specific data when determining
  which treatment will be more effective? Why? Draw a Paik diagram if it
  helps to make your point more clearly. \vspace{0px}
\item
  Discuss whether you should use the aggregate (marginal) or the
  segregated (conditional) data to attempt to determine the true effect
  in the following situation: There are two doctors in a small town.
  Each has performed 100 surgeries in his career, which are of two
  types: one very difficult surgery and one very easy surgery. The first
  doctor performs the easy surgery much more often than the difficult
  surgery and the second doctor performs the difficult surgery more
  often than the easy surgery. You need surgery, but you do not know
  whether your case is easy or difficult. Should you consult the success
  rate of each doctor over all cases, or should you consult their
  success rates for the easy and difficult cases separately, to maximize
  the chance of a successful surgery? Why? \vspace{0px}
\item
  Discuss whether you should use the aggregate (marginal) or the
  segregated (conditional) data to attempt to determine the true effect
  in the following situation: In a study of a group of male 50 to
  55-year-old long-time smokers, researchers compared a group of heavy
  smokers with matched group (same age range, sex and similar
  socioeconomic and environmental backgrounds) of light smokers. It was
  found that lung function was worse in the group of heavy smokers than
  in the group of light smokers. The researchers also measured the
  amount of tar deposit in the lungs of the subjects and classified
  subjects as having heavy or light tar deposits. Would you get a better
  indication of the effect of smoking by comparing the aggregated data
  for the two groups or by comparing the tar level specific data? Why?
  \vspace{0px}
\item
  When interpreting a study that purports to show a relationship between
  two variables, what do you think are the three most important
  questions that you should ask? Discuss as succinctly as you can the
  consequences of the answers to those questions. \vspace{0px}
\item
  R. A. Fisher insisted that causal inference was impossible in the
  absence of an experiment with random assignment to a `treatment'
  variable. Discuss why Fisher's position could be considered correct
  but why it may be considered impractical? \vspace{0px}
\item
  Give an example of a situation in which we would be interested in
  predictive inference and an example in which we would be interested in
  causal inference. \vspace{0px}
\item
  Here are some fictitious data on the rate of complications for
  appendectomies performed at University Hospital, a large urban
  teaching and research hospital, and in County Hospital, a small-town
  hospital: at University Hospital there were 800 cases with 100
  (12.5\%) resulting in complication and at County Hospital there were
  200 cases resulting in 5 (10\%) complications. The p-value for a test
  of the hypothesis that there is no difference in the rate at the two
  hospitals is 0.0037. Suppose that appendectomies can be classified as
  high risk or low risk and that the high risk cases tend to be directed
  disproportionately to University Hospital instead of County Hospital.
  Construct two hypothetical tables, one for each level of risk, and
  draw a Paik diagram that shows how it is possible for both high- and
  low-risk patients to have a lower probability of complications at
  University Hospital than at County Hospital, although, overall the
  probability of complications is higher at University Hospital than at
  County Hospital. \vspace{0px}
\item
  Suppose a test for glaucoma has a sensitivity of .95 and a specificity
  of .90. You receive the test as a routine test on a regular visit to
  your optometrist. The prevalence of glaucoma in your age, ethnic and
  gender group among people who have not been previously diagnosed is 1
  per 100. The test, alas, is positive. Use a natural frequency table to
  find the probability that you have glaucoma given the positive test
  result. \vspace{0px}
\item
  A group of major medical journals are now requiring that all authors
  who intend to publish in their journal must preregister their
  experimental designs and their intended analyses for all the clinical
  endpoints (responses) they intend to report before obtaining data if
  they intend to publish their results in their journal. Authors must
  also agree to publish their findings whether the results achieve
  statistical significance or not. In what ways does this policy
  contribute to mitigating lack of reproducibility? \vspace{0px}
\item
  A group of major medical journals are now requiring that all authors
  who intend to publish in their journal must preregister their
  experimental designs and their intended analyses for all the clinical
  endpoints (responses) they intend to report before obtaining data if
  they intend to publish their results in their journal. Authors must
  also agree to publish their findings whether the results achieve
  statistical significance or not. In what ways does this policy
  contribute to mitigating lack of reproducibility? \vspace{0px}
\item
  A study investigated whether there was a higher risk of complications
  when women gave birth at home with the assistance of a midwife instead
  of giving birth in a maternity ward in a hospital. Of 400 women who
  chose to give birth at home 20 had complications and of 2,000 women
  who gave birth in a hospital 200 had complications. Do you think that
  this is an experimental study or an observational study? Why?
  \vspace{0px}
\item
  (continued from the previous question) The data suggest that it is
  safer (in the sense of a lower rate of complications) to give birth at
  home than to give birth in the hospital. Discuss whether this implies
  that a woman should consider giving birth at home in order to reduce
  her risk of complications. Identify at least one plausible confounding
  factor and one plausible mediating factor that could partly explain
  the results of the study \vspace{0px}
\item
  (continued from the previous question) Choose a possible confounding
  factor and use a Paik diagram to show how controlling for this
  confounding factor could reverse the direction of association between
  the rate of complications and the location of birth: home or hospital.
  \vspace{0px}
\item
  This graph shows the mean verbal and mean math scores on the SAT test
  in each of the 50 states of the US.
  \newline\nopagebreak \includegraphics{m4939/verbal_math.PNG}
  \newline\nopagebreak Make an intelligent guess of the correlation
  between these two variables. Show the basis for your guess -- perhaps
  by drawing an appropriate geometric figure on the graph above.
  \vspace{0px}
\item
  Suppose you were to read about a study based on a random survey of
  Ontario medical records that shows that smokers have twice as high a
  risk of kidney disease as non-smokers. Is it reasonable to conclude
  that smoking causes a higher risk of kidney disease? Why or why not?
  \vspace{0px}
\item
  The following XKCD cartoon shows two statisticians interpreting the
  same data: one who uses a frequentist approach unquestioningly and one
  who uses a Bayesian approach.
  \newline\nopagebreak \includegraphics{m4939/explode.PNG}\newline\nopagebreak Make
  some reasonable assumptions, stating them explicitly, and calculate a
  reasonable value for the Bayesian statistician's posterior probability
  that the sun has exploded. Discuss why there is a difference between
  the `p-value' of 0.027 and the Bayesian posterior probability. Under
  what circumstances would you expect a p-value to be close to a
  posterior probability? \vspace{0px}
\item
  Which of the following R expressions result in the following output?
  \newline\newline \texttt{{[}1{]}\ \ 8} \newline\newline (Write `Y' for
  yes, `N' for no, and `D' or blank for `do not know'. +1 for a correct
  answer, -1 for a wrong answer and 0 for `D')\newline\newline \_\_\_\_
  \texttt{"+"(5,3)}\newline\newline \_\_\_\_
  \texttt{"/"(16,2)}\newline\newline \_\_\_\_
  \texttt{2\^{}3}\newline\newline \_\_\_\_
  \texttt{4\ +\ 4}\newline\newline \_\_\_\_ \texttt{"\^{}"(3,2)}
\end{enumerate}

\vspace{0px}

\begin{enumerate}
\def\labelenumi{\arabic{enumi}.}
\setcounter{enumi}{100}
\tightlist
\item
  Suppose we run this command: \newline\newline
  \texttt{a\ \textless{}-\ matrix(1:8,\ rnow\ =\ 2)} \newline\newline
  Then which of the following R expressions result in the following
  output? \newline\newline \texttt{{[}1{]}\ \ 8} \newline\newline (Write
  `Y' for yes, `N' for no, and `D' or blank for `do not know'. +1 for a
  correct answer, -1 for a wrong answer and 0 for `D')\newline\newline
  \_\_\_\_ \texttt{a(8)}\newline\newline \_\_\_\_
  \texttt{a{[}8{]}}\newline\newline \_\_\_\_
  \texttt{a(2,4)}\newline\newline \_\_\_\_
  \texttt{a{[}4,2{]}}\newline\newline \_\_\_\_ \texttt{a(2,4)}
\end{enumerate}

\vspace{0px}

\begin{enumerate}
\def\labelenumi{\arabic{enumi}.}
\setcounter{enumi}{101}
\item
  Suppose you have data on two variables, \(X\) and \(Y\), in each of
  \(J\) groups. Let \(\hat{\beta}_j\) represent the vector of
  coefficients from the least-squares regression ot \(Y\) on \(X\)
  within the \(j\)th group, \(j = 1, \ldots, J\). \textbf{Prove} that
  the inverse-variance weighted combination of the within-group
  estimated coefficients is the same as the vector of coefficients for
  the least-squares regression using the pooled data. Explain why (a
  sketch will suffice, no formal argument is needed) the pooled estimate
  of the slope estimates a combination of the within-group regression
  slope and the between-group regression slope. \vspace{0px}
\item
  Suppose you wish to estimate the relationship between income, \(Y\),
  and education, \(X\). Because of heteroscedasticity and curvature in
  the relationship you choose to fit a linear model using the log of
  \(Y\): \newline\newline
  \texttt{fit\ \textless{}-\ lm(\ log(Y)\ \textasciitilde{}\ X,\ data)}
  \newline\newline Write the R code you would use to plot the estimated
  increase in income associated with an extra year of education as a
  function of years of education. It is not necessary to include error
  bars in the plot. \vspace{0px}
\item
  Discuss situations when a) it would be important to include a variable
  that is not significant and b) it would be important to exclude a
  variable that is highly significant. Give examples of each situation.
  \vspace{0px}
\item
  Discuss how Lord's Paradox illustrates the usefulness of the gain
  score to test the difference between two treatments in which the
  response of interest has been measured before and after the
  application of the treatments but in which treatment assignment has
  not been randomized. Compare (a) regression of the gain score on the
  treatment variable with (b) regression of the post-test on the
  treatment variable using the pre-test as a covariate. Which of the two
  methods would be better if treatments had been randomly assigned?
  Discuss why. \vspace{0px}
\item
  Consider a mixed model of the form
  \texttt{lme(\ Y\ \textasciitilde{}\ X,\ data,\ random\ =\ \textasciitilde{}\ 1\ +\ X\textbar{}\ id)}
  in which there are two observations per cluster and the predictor,
  \texttt{X}, has the same two values, 0 and 1, in each cluster.
  Determine whether the variance parametrization of the model is
  identifiable. \vspace{0px}
\item
  A survey of Canadian families yielded average `equity' (i.e.~total
  owned in real estate, bonds, stocks, etc. minus total owed) of
  \$48,000. Aggregate government data of the total equity in the
  Canadian population shows that this figure must be much larger, in
  fact more than three times as large. Does this show that respondents
  must tend to dramatically underreport their equity or is there a
  probable explanation that is consistent with honest reporting by
  respondents? \vspace{0px}
\item
  The output below uses the schizophrenia data in which patients were
  observed at years 1, 2, 3, 4, 5, and 6 taking one of three drugs:
  Atypical, Clozapine, Typical each year.
  \newline\nopagebreak \includegraphics{m4939/drugs.PNG}\newline\nopagebreak Sketch
  the predicted response as a function of time (with years ranging from
  0 to 6), for a patient who took Atypical drugs 1/3 of the time and
  Clozapine 2/3 of the time if they were on Atypical drugs and if they
  were on Clozapine. On the graph, indicate the numerical value of the
  intercept and of the slope for each of the two lines. \vspace{0px}
\item
  Let Y and X be a numerical variables and let G be a factor. Consider
  the following models. All but one of these models will produce the
  same regression coefficient for X or Xr but they will produce
  different standard errors. Identify the model that produces a
  different coefficient. Rank the others where you can according to the
  standard error of the estimated coefficient for X stating which would
  be equal if any (assume a very large n and ignore the effect of slight
  differences in degrees of freedom for the error term). Explain your
  reasoning briefly. \newline\newline \null\quad (A)
  \texttt{Y\ \textasciitilde{}\ X\ +\ G} \newline \null\quad (B)
  \texttt{Y\ \textasciitilde{}\ X} \newline \null\quad (C)
  \texttt{Yr\ \textasciitilde{}\ Xr} where Yr is the residual of Y
  regressed on G and Xr is the same for X \newline \null\quad (D)
  \texttt{Y\ \textasciitilde{}\ Xr} \newline \null\quad (E)
  \texttt{Y\ \textasciitilde{}\ X\ +\ Xh} where Xh is the predictor of X
  in the regression of X on G \newline \null\quad (F)
  \texttt{Y\ \textasciitilde{}\ X\ +\ Xh\ +\ Zg} where Zg is a `G-level'
  numerical variable, i.e.~it has the same value for all observations
  within any value of G. \newline \vspace{0px}
\end{enumerate}

\end{document}
